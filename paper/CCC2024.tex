\documentclass[english]{cccconf}
%\documentclass[usemulticol,english]{cccconf}
\usepackage[comma,numbers,square,sort&compress]{natbib}
\usepackage{algorithm,algorithmic,amsmath,amssymb,epstopdf,nicematrix}

\begin{document}

\title{Fuel Feed Control for Asymmetric Arranged Multi-tanks Aircraft: A Hybrid MPC Approach}

\author{Haoyu Miao\aref{sustech},
        Mengfan Cao\aref{sustech},
        Yuning Tian\aref{sustech},
        Shibo Chen\aref{sustech},
        Zaiyue Yang\aref{sustech}}

\affiliation[sustech]{Southern University of Science and Technology, Shenzhen 518055, P.~R.~China
        \email{haoyumiao97@gmail.com, chensb@sustech.edu.cn, yangzy3@sustech.edu.cn}}

\maketitle

\begin{abstract}
Aircraft fuel system, which provides a continuous source of fuel to the engine, is an important component of the aircraft.
Although the sequential fuel feed strategy for the symmetric arranged multi-tanks aircraft is widely used in today's aircraft, the fuel feed strategy of the asymmetric arranged multi-tanks is still a challenge.
In this article, a two layer offline approach is developed to obtain the fuel feed strategy to minimize the difference between the actual center of gravity (CG) and the desired CG.
The performance of the proposed approach is tested in a case study based on the test data of aircraft pitch movement. 
The result indicates that the proposed approach solves the problem with an offline manner from the optimization perspective.

\end{abstract}

\keywords{asymmetric arranged multi-tanks, fuel feed control, hybrid MPC, warm start}

% Please remove or comment out the following line if the footnote is not necessary
\footnotetext{This work is supported by National Natural Science
Foundation (NNSF) of China under Grant 00000000.}

\section{Introduction}

This paper is organized as follows.
Section \ref{sec:related_work} reviews the related work of this study.
Section \ref{sec:system_and_problem} describes the system model and the problem formulation.
Section \ref{sec:hybrid_mpc} illustrates the proposed hybrid MPC method with the timer.
The experiment result is given in Section \ref{sec:numerical_examples}.
Finally, Section \ref{sec:conclusion} concludes this study as the proposed method can efficiently solve the fuel feed control problem with an online manner.

%\begin{figure}[!htb]
%  \centering
%  \includegraphics[width=\hsize]{fig1.eps}
%  \caption{The figure caption}
%  \label{fig1}
%\end{figure}

\section{Related Work}\label{sec:related_work}
\textit{MPC with minimum duration constarints.}
There are some work consider the minimum duration constraint in the MPC framework. 
Kamal et al. \cite{kamal2012control}, \cite{kamal2015traffic} considers the control of traffic signal in a MPC framework, in which the traffic signal whether red or green has a minimum duration, for example, 20 sec.
The minimum duration is considered as the step size in the prediction horizon and decision step in the control problem.


\section{System Model and Problem Formulation}\label{sec:system_and_problem}
The details of the model is refered to \cite{miao2021optimal}.

\subsection{System Model}

\begin{equation}\label{aircraft cg constraint}
\left\{
\begin{aligned}
x_c(t) &= \frac{\textbf{m}(t)^T\textbf{x}_{co}(t)}{M+\textbf{1}^T\textbf{m}(t)}\\
y_c(t) &= \frac{\textbf{m}(t)^T\textbf{y}_{co}(t)}{M+\textbf{1}^T\textbf{m}(t)}\\
z_c(t) &= \frac{\textbf{m}(t)^T\textbf{z}_{co}(t)}{M+\textbf{1}^T\textbf{m}(t)}
\end{aligned}
\right.
\end{equation}

\begin{equation}
  \mathbf{u}_{t|\tau}^T = \left[\mathbf{x}_{t|\tau}^T, \mathbf{y}_{t|\tau}^T\right]^T \in \mathbb{R}^{2n}
\end{equation}

$[T] = {0, 1, \cdots, T}$

Constraints:
\begin{equation}\label{eq:fuel_weight_dynamics}
\begin{aligned}
\textbf{m}_{0|\tau} &= \textbf{m}_{\tau}\\
\textbf{m}_{t+1|\tau} &= A\textbf{m}_{t|\tau}+ B\textbf{u}_{t|\tau}, t \in [T]\setminus T
\end{aligned}
\end{equation}

\begin{equation}\label{eq:weight_bound_constraint}
% 0\leq m_i(t) \leq a_ib_ic_i\rho,~ \forall i\in\mathcal{M},\forall t
\mathbf{0} \leq \mathbf{m}_{t|\tau} \leq \bar{\mathbf{m}}, ~\forall t \in [T]
\end{equation}

\begin{equation}\label{eq:feed_bound_constraint}
  \mathbf{0} \leq \mathbf{y}_{t|\tau} \leq \min \{K\mathbf{x}_{t|\tau},\bar{\textbf{y}}\}, ~\forall t \in [T]\setminus T
\end{equation}

The least fuel consumption constraint:
\begin{equation}\label{eq:least_fuel_consumption_constraint}
%\sum_{i\in \mathcal{M}_{m}}y_i(t) \geq h(t), ~\forall t \in [T]\setminus T
\mathbf{1}_{\mathcal{M}}^T\mathbf{y}(t)\geq h(t), ~\forall t
\end{equation}

The minimum duration constraint.
\begin{equation}\label{minimum_duration_constraint}
  [x_i(t+1) - x_i(t)]\times\left[\sum_{j=1}^L(x_i(t+j))-L\right]\geq 0
\end{equation}

\begin{equation}\label{continue constraint}
\begin{split}
    -M_1\left\{1-[x_i(t+1)-x_i(t)]\right\} \leq \sum_{j=1}^L(x_i(t+j))-L \\ \leq M_1\left\{1-[x_i(t+1)-x_i(t)]\right\}
\end{split}
\end{equation}


\subsection{Problem Formulation}
In the time slot $\tau$, the optimization problem to be solved is 
\begin{equation}
\begin{aligned}
  &\min_{\textbf{u}_{t|\tau}} \quad \sum_{t\in[T]} \lVert \textbf{c}_1(t) - \textbf{c}_2(t)\rVert_2^{2}\\
  & \begin{array}{r@{\quad}l@{}l@{\quad}l}
  s.t.& \textbf{m}_{0|\tau} = \textbf{m}_{\tau},\\
  & \textbf{m}_{t+1|\tau} = A\textbf{m}_{t|\tau}+ B\textbf{u}_{t|\tau}, t \in [T]\setminus T,\\
  & \mathbf{0} \leq \mathbf{m}_{t|\tau} \leq \bar{\mathbf{m}}, ~\forall t \in [T],\\
  & \mathbf{0} \leq V_{y}\mathbf{u}_{t|\tau} \leq \min \{K\cdot V_{x}\mathbf{u}_{t|\tau},\bar{\textbf{y}}\}, ~\forall t \in [T]\setminus T,\\
  & \mathbf{1}_{\mathcal{M}}^TV_{y}\mathbf{u}_{t|\tau}\geq h_{t|\tau}, ~\forall t \in [T]\setminus T,\\
  & V_{x}\mathbf{u}_{t|\tau} \in \{0,1\}^{n},~\forall t \in [T]\setminus T,\\
  & V_{y}\mathbf{u}_{t|\tau}\in \mathbb{R}_{+}^{n},~\forall t \in [T]\setminus T,\\
  & \text{constraints determined by timer.}
\end{array}.
\end{aligned}
\end{equation}



\section{Hybrid Model Predictive Control}\label{sec:hybrid_mpc}

To cope with the minimum duration constraint in model predictive control framework, we propose the MPC algorithm with timer.
The timer records the duration of each binary input.
If the duration of some binary input is larger than 0 but less than the minimum duration, the constraints determined by the timer will be added to the optimization problem.
If the duration reaches to the minimum duration, then the timer will be cleared to 0.

\begin{equation}\label{eq:timer_constraint}
  \mathbf{x}_{t|\tau}(i) = 1, \forall t \in [0, min\{T, L-timer(i)\}]
\end{equation}

The details of the algorithm is shown in Algorithm \ref{alg:hmpc}.

\begin{algorithm}
\caption{Hybrid Model Predictive Control}
\label{alg:hmpc}
\begin{algorithmic}[1]
\STATE Initialize model parameters, prediction horizon $T$
\STATE Initialize the timer $\mathbf{t}_{dura}$
\STATE Initialize current state $\mathbf{m}_0$
\WHILE{not at the end of the time horizon}
    \STATE Measure current state $\mathbf{m}_{\tau}$
    \STATE Compute optimal control sequence $\mathbf{u}_{t|\tau}^* = \text{HMPC}(\mathbf{m}_{\tau})$
    \STATE Apply the first control action $\mathbf{u} = \mathbf{u}_{0|\tau}^*$
    \STATE Update the timer $\mathbf{t}_{dura}$ 
    \STATE Simulate system dynamics using $\mathbf{u}$ 
    \STATE $\tau = \tau + 1$
\ENDWHILE
\end{algorithmic}
\end{algorithm}


\section{Numerical Examples}\label{sec:numerical_examples}
In the numerical experiments, we consider the example in the Fig.\ref{Fig.fuel_tank}. 
$\#2,\#3,\#4,\#5$ are main tanks, $\#1,\#6$ are backup tanks.
The geometric center coordinates of the fuel tanks are given in Table \ref{tab:geometric_center}. 
Due to the limitations of the aircraft structure, at most $2$ main tanks can feed the engine simultaneously,  and totally no more than $3$ tanks can feed simultaneously.
The proposed method is implemented with python on a computer with Intel$^\circledR$ Core\texttrademark ~ i7-8750H CPU @2.20GHz. 
The MIQP are solved with Gurobi 10.0.

\begin{equation}
\left\{
\begin{aligned}
m_{1,t+1|\tau} &= m_{1, t|\tau}-y_{1, t|\tau}\\
m_{2,t+1|\tau} &= m_{2, t|\tau} - y_{2, t|\tau} +y_{1, t|\tau}\\
m_{3,t+1|\tau} &= m_{3, t|\tau} - y_{3, t|\tau}\\
m_{4,t+1|\tau} &= m_{4, t|\tau} - y_{4, t|\tau}\\
m_{5,t+1|\tau} &= m_{5, t|\tau} - y_{5, t|\tau} +y_{6, t|\tau}\\
m_{6,t+1|\tau} &= m_{6, t|\tau} -y_{6, t|\tau}
\end{aligned}
\right.
\end{equation}

A is an identity matrix.

\begin{equation}
B = 
\left[\begin{array}{cccccc|cccccc}
  & & & & & & -1 & 0 & 0 & 0 & 0 & 0 \\
  & & & & & & 1 & -1 & 0 & 0 & 0 & 0 \\
  & & & & & & 0 & 0 & -1 & 0 & 0 & 0 \\
  & & & & & & 0 & 0 & 0 & -1 & 0 & 0 \\
  \multicolumn{6}{c|}{\raisebox{2ex}[0pt]{\Huge0}} & 0 & 0 & 0 & 0 & -1 & 1 \\
  & & & & & & 0 & 0 & 0 & 0 & 0 & -1 \\
\end{array}\right]
\end{equation}

\begin{table}[htbp]
\label{tab:geometric_center}
\caption{Geometric Center Coordinates of Fuel Tanks}
\begin{center}
\begin{tabular}{c|c|c|c}
\hhline
\textbf{Fuel}&\multicolumn{3}{|c|}{\textbf{Geometric Center Coordinates (Unit:m)}} \\
\cline{2-4} 
\textbf{Tank} & \textbf{\textit{x}}& \textbf{\textit{y}}& \textbf{\textit{z}} \\
\hline
\#1 & 8.913043& 1.20652174 & 0.61669004\\%$^{\mathrm{a}}$& &  \\
\#2 & 6.91304348 & -1.39347826 & 0.21669004 \\
\#3 & -1.68695652 & 1.20652174 & -0.28330996 \\
\#4 & 3.11304348 & 0.60652174 & -0.18330996\\
\#5 & -5.28695652 & -0.29347826 & 0.41669004\\
\#6 & -2.08695652 & -1.49347826 & 0.21669004\\
\hhline
%\multicolumn{4}{l}{$^{\mathrm{a}}$Sample of a Table footnote.}
\end{tabular}
\label{Table:Geometric Center}
\end{center}
\end{table}


\section{Conclusion}\label{sec:conclusion}
test

\bibliographystyle{unsrt}
\bibliography{references}


\end{document}

